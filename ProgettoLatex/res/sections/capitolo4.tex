\chapter{Organizzare le classi}

La relazione tra classi si può distinguere tra "è un" (\textit{is-a}) ed "ha un" (\textit{has-a}), ovvero ereditarietà di tipo e ereditarietà di implementazione. Quella di tipo si ha quando si estende effettivamente la classe e si sfrutta il polimorfismo, mentre quella di implementazione si ha quando è comodo riutilizzare il codice di un'altra classe.
\section{Classi astratte}
Si può creare una classe astratta con la keyword \texttt{abstract}. Questo permette di dichiarare metodi astratti (con \texttt{abstract}) che posso non definire. In compenso, non posso creare oggetti di quel tipo, ma si possono creare dei riferimenti che saranno sicuramente polimorfi.

Se una classe contiene un metodo astratto, la classe va marcata astratta o si ha un errore.
La classe astratta può avere dei campi dati; in questi casi viene definito un costruttore astratto che verrà richiamato dalle derivate.
Dei metodi concreti possono invocare metodi astratti che a runtime verranno invocati virtualmente nel modo corretto.
Una classe astratta non può avere metodi statici/final astratti perché richiedono un dispatching statico. \\
Una sottoclasse rimane astratta finché non definisce tutti i metodi astratti della base.

Di solito le classi astratte si usano per l'\textbf{ereditarietà di implementazione} perché permettono di scrivere una struttura riutilizzabile dalle derivate.

\section{Interfacce}
Le interfacce si usano per l'\textbf{ereditarietà di tipo}. Le interfacce definiscono funzionalità su un tipo senza indicare come è implementato quel tipo. Si dichiara con la keyword \texttt{interface Nome\{…\}} e ha regole specifiche:
\begin{itemize}
\item si può dichiarare solo campi marcati public static e final, e metodi astratti pubblici (abstract è implicito)
\item non può avere metodi statici (da Java 8 è possibile)
\item non sono istanziabili, neanche come sottooggetti, quindi non hanno il costruttore
\item può avere accessibilità public o package
\end{itemize}
Un'interfaccia non può essere istanziata, quindi l'unico modo per usare il tipo dell'interfaccia è implementandola con una classe che ne definisce \textbf{tutti} i metodi. La classe diventa sottotipo, quindi è possibile usare un riferimento dell'interfaccia in maniera polimorfa. La classe implementa con la keyword \texttt{implements}. 

Le interfacce permettono di implementare una sorta di "ereditarietà multipla" di tipo; infatti una classe può implementare più interfacce, ma estendere solo una classe. Può farlo contemporaneamente, prima \texttt{extends}, poi \texttt{implements} (\texttt{extends A implements B, C}).

Le interfacce possono estendersi tra loro, formando gerarchie; è possibile fare dei cast e instanceof. Non hanno un'interfaccia base comune, ma vale la relazione Interface$\le$Object. \\
\textbf{Relazione di sottotipo aggiornata}: T1$\le$T2 se vale
\begin{itemize}
\item T1 e T2 sono lo stesso tipo
\item T1 è definito come T1 extends T2, T1 implements T2
\item T1 e T2 sono tipi array, T1=A[ ], T2=B[ ] e A$\le$B
\item esiste un T3 tale che T1$\le$T3 e T3$\le$T2
\item T2 è di tipo Object, anche se T1 è interfaccia
\end{itemize}
Le conversioni cambiano leggermente. Se un riferimento ref ha come TS un'interfaccia, è sempre possibile convertire esplicitamente ref a qualsiasi classe, senza dare errori a compile time. Può però dare ClassCastException dinamicamente se TD(ref) non è $\le$T.

Le interfacce hanno un grosso problema: non è possibile aggiungere metodi e funzionalità successivamente perché invalidano tutte le classi che hanno implementato l'interfaccia. Da Java 8 esistono i metodi di default (\textbf{defender methods} \footnote{\url{https://dzone.com/articles/introduction-default-methods}}) che servono a risolvere questo problema. Questi metodi hanno un'implementazione di default della funzione così da permettere di far evolvere un'interfaccia senza invalidare le classi che l'hanno implementata. Se una classe eredita due implemetazioni del metodo di default, verrà scelta la scelta quella della classe concreta più specifica.

Da Java 8 è possibile creare metodi statici in un'interfaccia, ma non possono essere ridefiniti dai sottotipi (vanno definiti nella prima dichiarazione).

\section{Classi interne}
Le classi interne sono classi definite dentro ad altre classi, come membro (classi interne di istanza o statiche) o dentro un blocco di codice (classe interna locale). Un tipo innestato è parte del tipo che lo racchiude ed entrambe possono accedere a tutti i membri dell'altra classe (anche privati!).

Le classi interne possono essere private in modo da non essere accessibili dalle altre classi del package.
La classe interna non è sottotipo della classe contenitore. \\
\textbf{Nota per gli esempi}: Inner = tipo interno, Outer = tipo esterno.

\paragraph{Classe interna di istanza}
È considerata un membro, quindi ha un qualsiasi marcatore di accesso e può essere static. Se la classe non è statica, un oggetto della classe interna ha un riferimento "\texttt{outerThis}\footnote{Nota: non esiste letteralmente la keyword outerThis!}" all'oggetto della classe esterna che lo ha creato.

La classe interna può accedere a tutti i membri della classe esterna tramite l'\texttt{outerThis}, ovvero \texttt{Outer.this.membroEsterno}.
La classe contenitore può accedere a tutti i membri della classe interna tramite oggetti della classe interna.

Per costruire un oggetto di tipo Inner è necessario usare la classe esterna. Se la classe non è statica, si usa \texttt{this.new Inner()}, mentre se è statica \texttt{ref.new Inner()}, dove ref è di tipo Esterno.
Se la classe interna è accessibile dall'esterno, si può usare il tipo Inner anche al di fuori del contenitore tramite \texttt{nomeClasseEsterna.Inner}. 
Non contiene membri statici.
\vspace{2\parskip}
\begin{lstlisting}
Outer ref = new Outer();
// se non e' statica
Outer.Inner InnerObj = this.new Inner();
// se e' statica
Outer.Inner InnerObj = ref.new Inner();
\end{lstlisting}
La classe interna può implementare un'interfaccia; in particolare se fatto privato viene nascosta dall'esterno e l'implementazione delle classi esterne non dipende da quella dell'interfaccia. Allo stesso modo è possibile fare classi interne che estendono una classe, creando una sorta di ereditarietà multipla.

\paragraph{Classe interna statica} % ok
A volte non serve che la classe interna faccia riferimento ad un oggetto della esterna, quindi torna comodo dichiararla statica. In questo caso, non esiste l'\texttt{outerThis} e non serve un oggetto della classe esterna per crearne uno di quella interna. Però la classe interna non può accedere ai campi non statici della classe esterna.\\
Si definisce statica se non ha bisogno di accedere ai membri non statici della classe esterna: la classe esterna estende una classe D1, mentre la classe interna estende D2; in questo modo la classe esterna può accedere anche ai campi di D2.

\paragraph{Classe interna anonima}
Sono classi interne senza nome, di solito usate per implementare interfacce. Si possono creare oggetti di quella classe, ma non possono essere usati come tipi statici; non ha costruttori (a parte quello di default).
\begin{lstlisting}
public interface Interfaccia {...}

class C {
    // metodo che ritorna un oggetto di un sottotipo anonimo a Interfaccia
    public Interfaccia ritornaInterfaccia()
    {
        return new Interfaccia() {...}; // serve il punto e virgola!
    }
}
\end{lstlisting} 
\paragraph{Classi innestate in interfacce}
Anche le interfacce possono contenere classi e sono tutte statiche.

