\chapter{Programmazione Distribuita}
La programmazione distribuita entra in gioco quando si devono gestire processi su più macchine diverse. Le caratteristiche di un algoritmo distribuito sono:
\begin{itemize}
\item Concorrenza dei componenti
\item Mancanza di un \textit{Global Clock}, cioè l'ordine temporale non è strettamente condiviso
\item Fallimenti indipendenti tra loro
\end{itemize}
L'unica risorsa che condividono le macchine è la rete, attraverso la quale si scambiano messaggi e dati. 

\section{Socket TCP}
Due programmi possono comunicare attraverso gli stream associati ad una connessione di rete. Un tipo di comunicazione possibile è tramite i \textbf{socket TCP}. 

Ogni servizio fornito da un server è identificato da una porta logica; per gestire la comunicazione tra più client verso una stessa porta, viene usato un socket per ogni client. 

Il protocollo di comunicazione client/server è il seguente:
\begin{enumerate}
\item il server si mette in ascolto in attesa di richieste da parte di un client
\item un client richiede un servizio
\begin{itemize}
\item il client crea un socket per comunicare con il server utilizzando l'indirizzo dell'host del server e la porta del servizio richiesto
\item il server crea un socket per la comunicazione con il client richiedente
\end{itemize}
\item i due socket alle estremità vengono associati a degli stream per la comunicazione
\item una volta terminata la comunicazione, vengono chiusi i socket e la connessione 
\end{enumerate}
In particolare, la classe base è \texttt{Socket}, da cui deriva \texttt{ServerSocket} per la creazione dei socket del server. Il \texttt{ServerSocket} viene costruito passando come parametro il numero della porta.

La classe \texttt{InetServer} ha gli oggetti che rappresentano gli indirizzi IP. Questa classe è necessaria per la creazione dei socket da parte del client tramite il costruttore\footnote{Ndr: ovviamente ci sono anche costruttori con più parametri, sia per Socket, che per ServerSocket.}:
\begin{lstlisting}
public Socket(InetAddress address, int port) throws IOException
\end{lstlisting}
Con il metodo \texttt{accept()}, il ServerSocket accetta una richiesta di connessione e ritorna il relativo socket del client. Una volta connessi, è possibile ottenere gli stream collegati con i metodi \texttt{getInputStream()} e \texttt{getOutputStream()}. Questi stream sono thread-safe, ma un solo thread alla volta può scriverci e/o leggerci. Infine con il metodo \texttt{close()} viene chiuso il socket (di solito solo quello del client dato che il ServerSocket deve rimanere in attesa di altre richieste). 

La programmazione distribuita tramite socket è di basso livello.

\section{Datagram}
Un'alternativa alla comunicazione tramite socket è quella dei Datagram. Questo protocollo manda pacchetti di informazioni verso una o più destinazioni, senza garanzia di ricezione o di ordinamento in arrivo. La classe da utilizzare per ricevere i pacchetti è \texttt{DatagramPacket}; la classe è \texttt{final}, quindi non si può derivare. 

