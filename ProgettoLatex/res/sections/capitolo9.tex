\chapter{Programmazione Distribuita}
La programmazione distribuita entra in gioco quando si devono gestire processi su più macchine diverse. Le caratteristiche di un algoritmo distribuito sono:
\begin{itemize}
\item Concorrenza dei componenti
\item Mancanza di un \textit{Global Clock}, cioè l'ordine temporale non è strettamente condiviso
\item Fallimenti indipendenti tra loro
\end{itemize}
L'unica risorsa che condividono le macchine è la rete, attraverso la quale si scambiano messaggi e dati. 

\section{Socket TCP}
Due programmi possono comunicare attraverso gli stream associati ad una connessione di rete. Un tipo di comunicazione possibile è tramite i \textbf{socket TCP}. 

Ogni servizio fornito da un server è identificato da una porta logica; per gestire la comunicazione tra più client verso una stessa porta, viene usato un socket per ogni client. 

Il protocollo di comunicazione client/server è il seguente:
\begin{enumerate}
\item il server si mette in ascolto in attesa di richieste da parte di un client
\item un client richiede un servizio
\begin{itemize}
\item il client crea un socket per comunicare con il server utilizzando l'indirizzo dell'host del server e la porta del servizio richiesto
\item il server crea un socket per la comunicazione con il client richiedente
\end{itemize}
\item i due socket alle estremità vengono associati a degli stream per la comunicazione
\item una volta terminata la comunicazione, vengono chiusi i socket e la connessione 
\end{enumerate}
In particolare, la classe base è \texttt{Socket}, da cui deriva \texttt{ServerSocket} per la creazione dei socket del server. Il \texttt{ServerSocket} viene costruito passando come parametro il numero della porta.

La classe \texttt{InetServer} ha gli oggetti che rappresentano gli indirizzi IP. Questa classe è necessaria per la creazione dei socket da parte del client tramite il costruttore\footnote{Ndr: ovviamente ci sono anche costruttori con più parametri, sia per Socket, che per ServerSocket.}:
\begin{lstlisting}
public Socket(InetAddress address, int port) throws IOException
\end{lstlisting}
Con il metodo \texttt{accept()}, il ServerSocket accetta una richiesta di connessione e ritorna il relativo socket del client. Una volta connessi, è possibile ottenere gli stream collegati con i metodi \texttt{getInputStream()} e \texttt{getOutputStream()}. Questi stream sono thread-safe, ma un solo thread alla volta può scriverci e/o leggerci. Infine con il metodo \texttt{close()} viene chiuso il socket (di solito solo quello del client dato che il ServerSocket deve rimanere in attesa di altre richieste). 

La programmazione distribuita tramite socket è di basso livello.

\section{Datagram}
Un'alternativa alla comunicazione tramite socket è quella dei Datagram. Questo protocollo manda pacchetti di informazioni verso una o più destinazioni, senza garanzia di ricezione o di ordinamento in arrivo. 

La classe da utilizzare per ricevere o inviare i pacchetti è \texttt{DatagramSocket}. Non c'è distinzione di classe per la ricezione o l'invio. Un DatagramSocket si può connettere con il metodo \texttt{connect()} ad un indirizzo specifico, ma in questo modo i pacchetti vengono spediti o ricevuti solo verso/da quell'indirizzo. Con il metodo \texttt{send()} è possibile inviare un pacchetto rappresentato da un oggetto di \texttt{DatagramPacket}, mentre con \texttt{receive()} si riceve un pacchetto. La classe \texttt{DatagramPacket} è \texttt{final}, quindi non si può estendere. Infine il DatagramSocket va chiuso con il metodo \texttt{close()}.

\section{URL}
Esiste inoltre la possibilità di interagire direttamente con risorse web tramite la classe \texttt{final URL e URLConnection}. [...]\footnote{Ndr: non verrà approfondito.}

\section{Channel}
L'interfaccia \texttt{Channel} rappresenta un canale di I/O aperto o chiuso. L'interfaccia \texttt{NetworkChannel} invece è un Channel per un socket, cioè rappresenta una comunicazione su una rete. 
La classe \texttt{AsynchronousServerSocketChannel} è un canale asincrono basato su un server socket. [...]\footnote{Ndr: non verrà approfondito.}

\section{Modello RMI}
Il RPC (Remote Procedure Call) è un sistema per chiamare procedure di un'altra macchina rendendo la chiamata più simile possibile ad una chiamata locale. L'adattamento locale/remoto viene implementato da un componente detto \textbf{stub}. Questo meccanismo viene ripreso nei linguaggi orientati agli oggetti e prende il nome di RMI (Remote Method Invocation). Una chiamata RMI è composta dai seguenti passi:
\begin{enumerate}
\item il client chiama lo stub locale
\item lo stub prepara i parametri in un messaggio
\item lo stub invia il messaggio e il client viene bloccato
\item il server riceve il messaggio
\item il server controlla il messaggio e cerca l'oggetto chiamato
\item il server recupera i parametri e chiama il metodo destinatario
\item l'oggetto esegue il metodo e ritorna il risultato
\item il server prepara il risultato in un messaggio
\item il server invia il messaggio allo stub
\item lo stub riceve il messaggio di ritorno
\item lo stub recupera il risultato dal messaggio
\item lo stub ritorna al client il risultato
\end{enumerate}
Con l'evolvere della tecnologia, la strategia RMI viene usato poco e principalmente in ambienti piccoli e controllati.