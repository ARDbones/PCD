\chapter{Eccezioni}

\section{Tipi di eccezioni}
In Java, errori ed eccezioni sono oggetti, appartenenti a classi derivate da \textbf{Throwable}, la superclasse base. Il tutto si può distinguere tra eccezioni controllate e non controllate; le prime sono casi anomali gestibili, le seconde sono errori logici irrimediabili.
In particolare:
\begin{itemize}
\item \textbf{Error} è la sottoclasse degli errori severi/fatali, di solito \textbf{non} gestibili; terminano il programma (es: out of memory).
\item \textbf{RuntimeException} racchiude gli errori logici che avvengono runtime, \textbf{non} gestibili perché sono effettivamente errori logici che richiedono una modifica del codice (es: divisione per zero, nullpointer, ecc).
\item le rimanenti \textbf{Exception} vanno sempre gestite, o viene segnalato un errore a compile time.
\end{itemize}
Si può usare un'eccezione definita dall'utente creando una classe che estende \texttt{Exception}.

\section{Sollevare un'eccezione, blocchi try-catch, il costrutto finally}
Un'eccezione viene lanciata dal \texttt{throw} creando un oggetto del tipo eccezione che si vuole lanciare. La \texttt{throw} blocca l'esecuzione e cerca il codice che deve gestirla.

Se un metodo non gestisce l'eccezione al suo interno, è necessario indicare nella segnatura i tipi di eccezione che lancia.
\begin{lstlisting}
void m() throws Exc {
    try {throw new Exc("errore");}
}
\end{lstlisting}

Il codice che può sollevare eccezioni va in un blocco \texttt{try} e l'eccezione viene gestita dal primo \texttt{catch} di tipo compatibile. L'ordine delle clausole è da sottoclasse a superclasse, altrimenti la superclasse prende tutte le eccezioni. Se non si trova la \texttt{catch} adatta, l'eccezione viene rilanciata al metodo che ha invocato il metodo che ha lanciato l'eccezione; se non viene gestita, viene terminato il programma. Una \texttt{catch} non può catturare più eccezioni in relazione di tipo tra loro.
\begin{lstlisting}
catch(EccezioneBase B | EccezioneDerivata D){}
// Il compilatore segna un errore
\end{lstlisting}


Al termine del blocco try-catch è possibile inserire il blocco \texttt{finally}, un blocco di codice che viene sempre eseguito, anche se nel \texttt{try} è stato eseguito un return. È un blocco di codice protetto, che esegue sempre, ed è utile per rilasciare le risorse.

\section{Clausole throws e overriding di metodi}
Un metodo può non gestire l'eccezione al suo interno; in questo caso è necessario indicarlo nella segnatura con \texttt{throws Eccezione}.
Quando si hanno metodi ridefiniti con \texttt{throws} nella segnatura, il metodo ridefinito deve lanciare eccezioni che sono sottotipo del precedente e non deve avere più eccezioni. 

\section{RuntimeExceptions comuni}
\begin{itemize}
\item \texttt{ArithmeticException}: divisione per zero;
\item \texttt{NullPointerException}: riferimento nullo;
\item \texttt{ClassCastException}: cast dinamico fallito;
\item \texttt{ArrayIndexOutOfBoundsException}: tentativo di accedere ad un elemento oltre i limiti di un array.
\end{itemize}

\section{try-with-resources}
Prima di JSE7 le gestione delle risorse ed eccezioni era gestita attraverso il meccanismo \texttt{try/catch/finally}. Richiede una disciplina d'uso di gestione eccezioni innestate che non \`e immediata con spesso codice ridondante e ripetitivo per gestire la chiusura delle risorse che non servono. Java SE7 introduce una nuova funzionalit\`a chiamata \texttt{try-with-resources}.
 
\begin{lstlisting}
void printFileJava7() throws IOException {
	try(FileInputStream in = new FileInputStream("f.txt")) {
		int data = in.read();
		while(data != -1){
			System.out.print((char) data);
			data = in.read();
		}
	}
}
\end{lstlisting}

Quando il blocco try finisce (normalmente o lanciando un eccezione) le risorse saranno chiuse/rilasciate in automatico. Se un eccezione viene lanciata sia all'interno del blocco \texttt{try} che in chiusura/rilascio della risorsa, l'eccezione in chiusura verr\`a soppressa mentre quella interna al blocco lanciata al chiamante.

E' possibile usare il costrutto try-with-resources con tutte le risorse che implementano l’interfaccia \texttt{java.lang.AutoClosable}. Gli I/O stream visti implementano questa interfaccia.

E' inoltre possibile catturare molteplici errori in un unico blocco \texttt{catch}.

\begin{lstlisting}
try {
	// crea connessione db ed esegui query.
} catch(SQLException | IOException e) {
	...
} catch(Exception e) {
  ...
}
\end{lstlisting}

Attenzione: Non si possono catturare all'interno dello stesso catch eccezioni che sono in una relazione di tipo. E.g., non si possono inserire sia \texttt{ExceptionA} che \texttt{ExceptionB} se \texttt{ExceptionB} eredita sia direttamente che indirettamente da \texttt{ExceptionA}. Il compilatore si lamenta: \texttt{The exception ExceptionB is already caught by the alternative ExceptionA}
