\chapter{Stream, File e Lambda expressions}

\section{Stream e \texttt{java.io}}
In Java le operazioni di input e output sono trattate considerando i dati come flusso (stream). Lo stream é un'astrazione dei dispositivi di input/output: si apre lo stream, si leggono/scrivono i dati, si chiude lo stream.

In Java gli stream sono istanze di classi raccolte nel package \texttt{java.io}. Il pacchetto contiene due gerarchie di classi in base al tipo di flusso:
\begin{itemize}
 \item flusso di caratteri, usati per la lettura/scrittura di testo
 \item flusso di byte, usati per la trasmissione di dati in codice binario
\end{itemize}
Alla base delle gerarchie si trovano delle classi astratte con le operazioni di base:

\begin{lstlisting}
class Reader{
	int read()
	int read(char[] buff)
	abstract int read(char[] buff, int off, int cnt)
	int skip(long count)
	abstract void close()
	boolean ready()
}

class Writer{
	void write(char b)
	void write(String r)
	abstract void write(char[] buff, int off, int cnt)
	abstract void close()
	abstract void flush()
}

class InputStream{
	abstract int read()
	int read(byte[] buff)
	int read(byte[] buff, int off, inf cnf)
	int skip(long count)
	void close()
	boolean available()
}

class OutputStream{
	abstract void write(byte b)
	void write(byte[] buff)
	void write(byte[] buff, int off, int cnt)
	void close()
	void flush()
}
\end{lstlisting}

Alcuni gruppi di classi che estendono le precedenti:
\begin{itemize}
\item flussi \texttt{Filter}, che hanno un qualche filtro
\item flussi \texttt{Buffered}, che aggiungono la presenza di un buffer per ridurre gli accessi ai dispositivi di input/output
\item flussi \texttt{Piped}, che sono coppie di flussi in cui si legge in uno e si scrive nell'altro
\end{itemize}
I flussi standard (tastiera e video), sono flussi di byte.

\section{Stream e file}
Quando si trasmette con un file, serve la classe adeguata. Per i caratteri si usa \texttt{FileReader} e \texttt{FileWriter}, mentre per i byte si usa \texttt{FileInputStream} e \texttt{FileOutputStream}.
L'accesso al file può essere sequenziale o diretto.

\section{Serializzazione}
I dati trasmessi in input/output a volte possono essere proprio degli oggetti. Con \texttt{ObjectInputStream e ObjectOutputStream} è possibile rappresentare flussi di oggetti. 

Non tutti gli oggetti vanno bene: solo quelli che implementano l'interfaccia \texttt{Serializable}. L'interfaccia non ha metodi e serve solo a marcare che gli oggetti possono essere trasmessi tramite stream. Viene lanciata l'eccezione \texttt{(NotSerializableException} se si usa un oggetto che non implementa Serializable.

La serializzazione è il processo che trasforma un oggetto in un'opportuna sequenza di byte (\texttt{writeObject(obj)}) e viceversa (\texttt{readObject()} - deserializzazione). La deserializzazione ritorna un oggetto di tipo \texttt{Object}, quindi sta al programmatore fare un cast corretto in base al tipo adatto. 

Non tutti i tipi sono serializzabili. Tuttavia quando si serializza un oggetto, devono essere serializzati tutti i campi, soprattutto se sono riferimenti ad altri oggetti che però possono essere non serializzabili. In questi casi si può implementare la serializzazione custom:
\begin{itemize}
\item la classe deve avere il costruttore senza argomenti
\item la classe deve implementare i due metodi di lettura/scrittura
\item i campi non serializzabili vanno indicati con la keyword \texttt{transient}
\end{itemize}

\section{Espressioni Lambda}
Un'espressione Lambda in Java è un metodo senza dichiarazione esplicita con una forma \texttt{(parametri) -> \{corpo\}}.
Regole generiche:
\begin{itemize}
\item i parametri possono essere più di uno, separati da virgola, ma anche nessuno
\item le parentesi dei parametri si possono omettere se è un solo parametro
\item il tipo dei parametri può essere implicito o esplicito
\item le parentesi del corpo possono venire omesse se è un solo statement
\end{itemize}

Le espressioni Lambda vengono tradotte dal compilatore in funzioni di \textbf{interfacce funzionali}. Le interfacce funzionali contengono un singolo metodo astratto \textit{il quale è la traduzione dell'espressione} (incerto). Esistono delle interfacce già definite come:
\begin{itemize}
\item \texttt{Function} - accetta un argomento di tipo T e ritorna un tipo R
\item \texttt{Supplier} - nessun argomento e ritorna un tipo R
\item \texttt{Consumer} 
\item \texttt{Predicate} - un solo argomento e ritorna boolean 
\end{itemize}

\section{Stream API}

