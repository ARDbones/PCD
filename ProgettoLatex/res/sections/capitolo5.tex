\chapter{Generics e Collezioni}

\section{Generics}
In Java sono presenti i template di funzione e di classe, e si chiamano funzioni/classi generiche. La definizione utilizza un tipo parametrico T che deve essere specificato quando si vuole usare un'istanza di quel template\footnote{Ndr: ognivolta che verrà usato "template" si intende un qualsiasi Generic.} di classe/interfaccia, mentre per i metodi viene dedotto dal compilatore. Il tipo T non può essere un primitivo; vengono accettati solo tipi classe e interfaccia.
Per definire un generic si usano le parentesi angolari:
\begin{lstlisting}
public class Classe<T,Z> { }

public interface Interfaccia<T> { }

public <T> void f(T[] array) { }
\end{lstlisting}

Un metodo statico di una classe generica che utilizza un tipo parametrico deve essere un metodo generico o si ha un errore di compilazione. 
\begin{lstlisting}
public class Classe<T>{
	public static <T> void m(T[] arg) { }
}
\end{lstlisting}

Se un generic viene esteso o implementato, deve essere indicato il tipo esplicitamente nella segnatura.

\subsection{Vincoli sui tipi delle variabili e Wildcards}
È possibile porre dei vincoli sui tipi accettabili dal template per evitare errori logici (es: String in un template che confronta numeri). I vincoli sono espressi all'interno delle parentesi angolari indicando quali classi estende:

\begin{lstlisting}
public static <T extends C> min(T[ ] arg){ }
// dove C e' la classe/interfaccia base
\end{lstlisting}
Nell'esempio è sensato usare l'interfaccia \texttt{Comparable} al posto di C. \\
Se si vuole che T sia sottotipo di più tipi, si usa \texttt{<T extends A \& ... \& C>}. Nell'elenco devono andare prima la classe, poi le interfacce.

Un altro modo per porre dei vincoli e l'utilizzo di \textbf{wildcards}, cioè dei caratteri jolly che hanno un significato implicito.

\begin{table}[H]
\centering
\begin{tabular}{|c|c|c|}
\hline
\textbf{Nome} & \textbf{Sintassi} & \textbf{Significato} \\
\hline
upper bounded & ? extends B & qualsiasi sottotipo di B \\
\hline
lower bounded & ? super B & qualsiasi supertipo di B \\
\hline
unbounded & ? & qualsiasi tipo \\
\hline
\end{tabular}
\end{table}

\subsection{Generics e Ereditarietà}
Se un tipo è l'estensione di un altro, non comporta che lo siano anche le classi generiche istanziate. 
\begin{lstlisting}
class Dirigente extends Dipendente {...}
class Agente extends Dipendente {...}
ArrayList<Dirigente> a = new ArrayList<Dirigente>();
ArrayList<Dipendente> b = a; // NON FUNZIONA
Dipendente dip = new Agente();
b.add(dip); // OK
\end{lstlisting}
Ad esempio non è possibile usare un \texttt{ArrayList<Double>} al posto di un \texttt{ArrayList<Object>}, anche se Double$\le$Object.

\subsection{Type erasure}
La JVM attua il \textit{type erasure}, ovvero per ogni generic sostituisce il tipo parametrico con il primo bound o con Object se è unbounded. In questo modo le verifiche avvengono a compile time. A causa del type erasure di Java non è possibile:
\begin{itemize}
\item costruire oggetti che sono istanze del tipo parametro - \texttt{new T()}
\item costruire un array del tipo parametro - \texttt{new T[k]}
\item usare il tipo di una classe generica per definire campi dati statici, metodi statici o classi
\item fare overloading di metodi dove cambia solo il tipo parametrico, in quanto la \textit{type erasure} fa sì che i metodi abbiano la stessa signature
\item l'uso dell’operatore \texttt{instanceof (A instanceof T)}
\end{itemize}

\section{Collezioni}
Le collezioni in Java corrispondono ai contenitori, dove un oggetto rappresenta un insieme di oggetti. Mentre gli array in Java hanno una lunghezza fissa, un tipo \texttt{Collection} è un array dinamico di \texttt{Object}.

Esistono delle apposite collezioni per la concorrenza introdotte da Java 7 che sono considerate \textit{thread-safe}.

\subsection{Interfaccia List}
Rappresenta una sequenza di elementi. 
\paragraph{ArrayList}
Rappresenta un array ridimensionabile. Non è thread-safe. \\ Alcune operazioni:
\begin{itemize}
\item \texttt{size}
\item \texttt{isEmpty}
\item \texttt{get}
\item \texttt{set}
\item \texttt{itarator}
\item \texttt{listIterator}
\item \texttt{add}
\item \texttt{contains}
\item \texttt{indexOf} - simile a contains, ma ritorna l'indice dell'elemento (se presente)
\item \texttt{remove}
\end{itemize}
Costruttori:
\begin{itemize}
\item \texttt{ArrayList()} - lista con capacità iniziale a 10;
\item \texttt{ArrayList(Collection c)} - costruisce una lista contenente gli elementi della collezione passata;
\item \texttt{ArrayList(int capacità)} - costruisce una lista vuota di con una certa capacità iniziale.
\end{itemize}
\textbf{Nota bene}: la capacità non implica che venga creato un oggetto nullo! Quindi se cerco di ottenere un elemento con il \texttt{get(i)} viene lanciata l'eccezione \texttt{IndexOutOfBoundsException} se \textbf{i} non corrisponde ad un elemento inizializzato.

\paragraph{LinkedList}
Rappresenta una lista doppiamente linkata. Non è thread-safe. Presenta le stesse operazioni di \texttt{ArrayList} e alcune operazioni in più:
\begin{itemize}
\item \texttt{offer} - aggiunge un elemento in coda;
\item \texttt{peek} - ritorna il primo elemento della lista;
\item \texttt{pop} - ritorna e rimuove il primo elemento della lista;
\item \texttt{push} - inserisce un elemento in fronte.
\end{itemize}
Costruttori:
\begin{itemize}
\item \texttt{LinkedList()} - lista con capacità iniziale a 10;
\item \texttt{LinkedList(Collection c)} - costruisce una lista contenente gli elementi della collezione passata.
\end{itemize}

\subsection{Interfaccia Map}
Rappresenta un array associativo chiave-valore.
\paragraph{HashMap}
Insieme di coppie chiave-valore non ordinati. Non è thread-safe. Alcune operazioni:
\begin{itemize}
\item \texttt{clear}
\item \texttt{clone}
\item \texttt{containsKey}
\item \texttt{containsValue}
\item \texttt{entrySet}
\item \texttt{keySet}
\item \texttt{get}
\item \texttt{put}
\item \texttt{size}
\item \texttt{remove}
\end{itemize}
Costruttori:
\begin{itemize}
\item \texttt{HashMap()} - costruisce una mappa con capacità iniziale a 16 e fattore di carico a 0.75;
\item \texttt{HashMap(int capacità)} - costruisce una mappa vuota di con una certa capacità iniziale e fattore di carico 0.75;
\item \texttt{HashMap(int capacità, float carico)} - costruisce una mappa vuota di con una certa capacità iniziale e un certo carico;
\item \texttt{HashMap(Map m)} - costruisce una mappa contenente gli elementi della collezione passata;
\end{itemize}
\textbf{Nota}: a volte può essere essere utile ridefinire \texttt{hashCode()} nel caso di utilizzo di chiavi custom.

\paragraph{TreeMap}
Insieme di coppie chiave-valore ordinati secondo la chiave tramite un albero rosso-nero. Non è thread-safe. 

\subsection{Interfaccia Set}
Collezione di elementi che non può contenere duplicati.

\paragraph{HashSet}
Implementa un Set sfruttando le caratteristiche di un HashMap. Non è thread-safe. Alcune operazioni:
\begin{itemize}
\item \texttt{add}
\item \texttt{clear}
\item \texttt{clone}
\item \texttt{contains}
\item \texttt{isEmpty}
\item \texttt{size}
\item \texttt{remove}
\end{itemize}
Costruttori:
\begin{itemize}
\item \texttt{HashSet()} - costruisce un insieme vuoto con capacità iniziale a 16 e fattore di carico a 0.75;
\item \texttt{HashSet(int capacità)} - costruisce un insieme vuoto di con una certa capacità iniziale e fattore di carico 0.75;
\item \texttt{HashSet(int capacità, float carico)} - costruisce un insieme vuoto di con una certa capacità iniziale e un certo carico;
\item \texttt{HashSet(Collection c)} - costruisce un insieme contenente gli elementi della collezione passata;
\end{itemize}

\paragraph{TreeSet}
Implementa un Set ordinato secondo l'ordine naturale degli elementi. Sfrutta l'albero rosso-nero. Non è thread-safe. Alcune operazioni:
\begin{itemize}
\item \texttt{add}
\item \texttt{remove}
\item \texttt{ceiling}
\item \texttt{first}
\item \texttt{higher}
\item \texttt{last}
\item \texttt{subSet}
\end{itemize}
Costruttori:
\begin{itemize}
\item \texttt{TreeSet()} - costruisce un insieme vuoto ad albero;
\item \texttt{TreeSet(Comparator comp)} - costruisce un insieme vuoto di con una certa capacità iniziale e fattore di carico 0.75;
\item \texttt{TreeSet(SortedSet s)} - costruisce un insieme vuoto con ordine ed elementi dell'insieme passato;
\item \texttt{TreeSet(Collection c)} - costruisce un insieme ad albero contenente gli elementi della collezione passata;
\end{itemize}

\subsection{Classe Collections}
\texttt{Collections} è una classe che ha solo metodi statici e generici per agire sulle collezioni. Alcuni metodi\footnote{Ndr: la segnatura dei seguenti metodi è stata semplificata; leggere la documentazione Java per sapere la segnatura completa.}:
\begin{itemize}
\item \texttt{boolean addAll(Collection c, T... elements)} - aggiunge alla collezione gli elementi (più di uno);
\item \texttt{int binarySearch(List L, T key)} - cerca nella lista l'argomento key; 
\item \texttt{void fill(List L, T obj)} - rimpiazza tutti gli elementi della lista con l'elemento obj;
\item \texttt{void sort(List L)} - ordina gli elementi in ordine ascendente;
\item \texttt{void copy(List Out, List In)} - copia gli elementi dalla lista \textbf{In} alla lista \textbf{Out}.
\end{itemize}


\subsection{Classe Arrays}
La classe \texttt{Arrays} offre metodi per operare su strutture lineari (array e liste). Alcuni metodi\footnote{Ndr: vedi nota sopra.}:
\begin{itemize}
\item \texttt{List asList(T... a)} - ritorna una lista dall'array passato;
\item \texttt{int binarySearch(...)} - ricerca l'oggetto key nel range specificato; 
\item \texttt{T[] copyOf(T[] originale, int newLength)} - per ridimensionare un array;
\item \texttt{String deepToString(Object[] a)} - un \texttt{toString} sui contenuti; 
\item \texttt{boolean deepEquals(Object[] a, Object[] b)} - un \texttt{equals} sui contenuti; 
\item \texttt{void sort(Object[] a)} - ordina gli elementi. 
\end{itemize}
