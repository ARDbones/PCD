\chapter{Struttura dei programmi Java}

\section{Java e il Main}
Il linguaggio Java è orientato completamente agli oggetti, quindi non ci sono funzioni, solo metodi di classe. Anche il \texttt{main} è un metodo di classe, \textbf{statico} (non ha oggetto di invocazione), \textbf{pubblico} (accessibile da chiunque), con argomento di tipo \texttt{String[ ]} e senza valore di ritorno (\textbf{void}).
Un programma deve avere una classe pubblica con il metodo main, invocato automaticamente all'inizio dell'esecuzione.
\begin{verbatim}
public static void main(String[ ] args){ }
\end{verbatim}

\section{Pacchetti - Package}
Un file \textit{.java} contiene una classe \textbf{pubblica} che ha lo stesso nome del file. Le classi correlate tra loro stanno nello stesso pacchetto. Un pacchetto evita il conflitto di nomi e serve a partizionare i moduli di un programma.

Ogni file che appartiene ad un pacchetto deve avere la dichiarazione come prima istruzione.
\begin{verbatim}
package nomepacchetto;
\end{verbatim}
Il nome completo di una classe è quindi \textbf{nomepacchetto.nomeclasse}.
Si può creare una gerarchia di pacchetti, ma non da privilegi di accesso tra i pacchetti.

Per usare in una classe un tipo dello stesso pacchetto basta il nome della classe, mentre se si usa un tipo esterno serve il nome completo o si "include" il pacchetto che lo contiene (\texttt{import pacchetto.*;}) o si utilizza la "dichiarazione d'uso" (\texttt{import pacchetto.classe;}). A differenza di C++, i pacchetti non vengono incluse fisicamente, quindi non influenzano le prestazioni.

\section{Operatori logici}
La valutazione degli operatori logici condizionali avviene per short-circuiting, cioè evitando di controllare il secondo operando se possibile. In particolare:
\begin{itemize}
\item \texttt{w \&\& z} - se w valuta a \textbf{false} allora z non viene valutato;
\item \texttt{x || y} - se x valuta a \textbf{true} allora y non viene valutato.
\end{itemize}

\section{Costrutto \textit{enhanced for} - il for migliorato}
In Java esiste il for migliorato (o anche foreach) che permette di semplificare le iterazioni di array e collezioni. Si usa ponendo tra parentesi il \texttt{tipo\_contenuto : nome\_contenitore}. Il costrutto itera automaticamente finché sono presenti valori nel contenitore.

\begin{lstlisting}
int[] array = {1, 2, 3, 4, 5};
for(int valore : array)
{
	// fai qualcosa con i valori, es:
	System.out.println(valore);
}
\end{lstlisting}